\documentclass{ltjsarticle}
\usepackage{amsmath}
\begin{document}
  \begin{table}[htbp]
    \caption{春学期時間割}
    \centering
    \begin{tabular}{c|c|c|c|c}
      Mon & Tus & Wed & Thu & Fri \\ \hline\hline
      - & - & (デジタル基礎) & - & - \\ \hline
      物理情報数学B & - & - & 物理情報数学A & (オプティクス) \\ \hline
      - & 理工学基礎実験   & - & - & - \\ \hline
      電気回路同演習 & 理工学基礎実験 & - & - & 電磁気学同演習 \\ \hline
      - & - & - & - & - \\
    \end{tabular}
  \end{table}
  \begin{table}[htbp]
    \caption{秋学期時間割}
    \centering
    \begin{tabular}{c|c|c|c|c}
      Mon & Tus & Wed & Thu & Fri \\ \hline\hline
      - & (計測光学) & - & (応用確率論) & - \\ \hline
      熱物理/量子力学入門& (分布系の数理) & - & - & 熱物理/量子力学入門 \\ \hline
      - &- & - & - & - \\ \hline
      (応用電磁気学同演習)& プログラミング基礎同演習 & - & - & (物性物理基礎)\\ \hline
      (生体計測)& - & - & - & - \\
    \end{tabular}
  \end{table}
  履修上の注意
  \begin{itemize}
    \item かっこでくくられた科目から8単位以上選びましたか?
    \item 一般教養の授業は第1学年と合わせて10単位以上取得しましたか?
    \item 英語のクラスと時限は確認しましたか?
  \end{itemize}
\end{document}